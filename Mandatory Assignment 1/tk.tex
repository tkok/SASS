\begin{equation}
\begin{split}
{\rm SYSTEM_1} \stackrel{\mathrm{def}}{=} (v\,talk_i , switch_i , give_i , alert_i : i = 1,2) \\
({\rm CAR}(talk_1 , switch_1) | {\rm BASE_1} | {\rm IDLEBASE_2} | {\rm CENTRE_1})
\end{split}
\end{equation}

\begin{equation}
\begin{split}
{\rm CENTRE_1} \stackrel{\mathrm{def}}{=} \overline{give_1} \langle talk_2 switch_2 \rangle .alert2.{\rm CENTRE_2} \\ 
{\rm CENTRE_2} \stackrel{\mathrm{def}}{=} \overline{give_2} \langle talk_1 switch_1 \rangle .alert1.{\rm CENTRE_1}
\end{split}
\end{equation}

\section{Exercise 2, p. 13}

\begin{equation}
\begin{split}
SYSTEM_1 \equiv  (v\overleftarrow{c})(CAR/talk_1,switch_1) | BASE_1 | IDLEBASE_2 | CENTRE_1) \\
\rightarrow (v\overleftarrow{c})(CAR(talk_1, switch_1) | \overline{switch_1}talk_2switch_2.IDLEBASE_1 \\
| IDLEBASE_2 | alert_2.CENTRE_2) \\
\rightarrow (v\overrightarrow{c}((CAR(talk_2,switch_2) | IDLEBASE_1 \\
| IDLEBASE_2 | alert_2.CENTRE_2) \\
\rightarrow (v\overrightarrow{c})(CAR(talk_2,switch_2) | IDLEBASE_1 | BASE_2 | CENTRE_2) \\
\equiv SYSTEM_2
\end{split}
\end{equation}

slut exercise 2