% Example LaTeX document for GP111 - note % sign indicates a comment

\documentclass[11pt]{article}
\usepackage{amssymb,amsmath}
% Default margins are too wide all the way around.  I reset them here
\setlength{\topmargin}{-.5in}
\setlength{\textheight}{9in}
\setlength{\oddsidemargin}{.125in}
\setlength{\textwidth}{6.25in}

\begin{document}
\title{Mandatory Assignment 1}
\author{Group 1\\
IT University of Copenhagen}
\renewcommand{\today}{September 22, 2012}
\maketitle
This assignment is a part of the Advance Mobile and Distributed Systems Seminar.

\section {Exercise 1, p. 12}

\begin{equation}
\begin{split}
SYSTEM_1 \stackrel{\mathrm{def}}{=} (v\,talk_i , switch_i , give_i , alert_i : i = 1,2) \\
(CAR(talk_1 , switch_1) | Base_1 | IDLEBASE_2 | CENTRE_1)
\end{split}
\end{equation}

\begin{equation}
IDLEBASE(t, s, g, a) \stackrel{\mathrm{def}}{=} a.BASE(t,s,g,a)
\end{equation}

\begin{equation}
BASE_i \stackrel{\mathrm{def}}{=} BASE(talk_i , switch_i , give_i , alert_i) \quad (i=1,2)
\end{equation}

\begin{equation}
\begin{split}
CENTRE_i \stackrel{\mathrm{def}}{=} \overline{give_1}talk_2switch_2.alert2.CENTRE_2 \\ 
\overline{give_2}talk_1switch_1.alert1.CENTRE_1 
\end{split}
\end{equation}






\end{document}
\end{verbatim}
\end{document}
